\paragraph{Improving Image Storage}
\section{Limitations and Future Work}
%\subsection{Limitations}
A limitation of our current implementation is the cost of evaluating custom scan configurations.
Our na\"{i}ve implementation takes days to process 24,988 images on 8 cluster nodes (12 cores/12 threads per node) with Westmere-class CPUs. 
We suspect that this time can drastically reduced without sacrificing significant bandwidth savings by aggressively pruning the search space or applying machine learning techniques to choose scan configurations.
Note that while the customization process for progressive JPEG is currently expensive, it only needs to be performed once, at write time. 

The PSNR metric is limited in its relevance to perceived visual quality~\cite{wang2002no}.
We often found that the PSNR of higher resolution resizes was higher than that of lower resolution resizes even with less image data read---this issue may be mitigated with conservative PSNR thresholds.
To the best of our knowledge, there is no standard, widely used method of computing the image quality of a resized image derived from a source image.
%Ideally, an image quality metric would not describe the similarity or difference of two images but be computable without a reference image.
%Such a scheme would resemble a typical use case more closely as users of an image hosting service do not need a reference image to evaluate image quality.

Finally, an issue when using progressive JPEG for dynamic resizing is the \emph{minimum} resolution of the resized images.
Progressive JPEG is less efficient in terms of read bandwidth for resizes smaller than 10\% of the source image, which may limit savings when the source images are much higher in resolution than their resized versions.
This threshold is due to JPEG's use of $8\times8$ macroblocks: even a single frequency coefficient represents at least $\frac{1}{64}$ of the total image data.
Even when approximations are used, this approach may require more read bandwidth than pre-resized images.
Still, using progressive JPEG should be more space-efficient than baseline JPEG for dynamic resizing.
%\paragraph{Future Work}

We expect that a solution to reduce the cost of enumerating custom JPEG scan configurations will be to prune the search space to a much smaller subset of likely ``good'' configurations.
It may be possible to obtain comparable results by only trying a few custom scan
configurations per image---with this subset being determined by identifying the best configurations when na\"{i}vely re-encoding a larger dataset of images.
Along these lines, even choosing from a larger pool of configurations may be tractable if a machine learning model is applied to each image to choose the best configuration.


\paragraph{Data Augmentations for Pretraining}
Data augmentations are an interesting topic of study in unsupervised or semi-supervised learning settings.
From one perspective, a reasonable objective is to use augmentations to enforce consistency between perturbed input examples originating from the same source.
From another, augmentations that distort images can be potentially useful in creating pre-training tasks (e.g., correctly ordering shuffled image patches, reorienting a rotated image).
However, not all augmentations appear to be useful as components of pre-training tasks, as it may be desirable for the downstream fine-tuned model to normalize away such augmentations.
In these situations, it may be useful to enforce that neural networks remain equivariant to augmentation attributes, just as convolution has been classically motivated as a translation equivariant operator.


\paragraph{Solving the Scale Equivariance Problem}
Another approach to solving the scale invariance problem is to change the region of interest that is used as input to a computer vision model dynamically, depending on the positioning and scale of an object in a frame.
This approach would effectively be an extension of neural network models that use a version of the attention mechanism popular in natural language processing models.
While convolution-based architectures are currently more common in computer vision, recent work has shown that with sufficient computation, attention-based architectures can also be competitive with state-of-the-art convolution approaches.


\paragraph{Tuning Deep Learning Kernels}
While tremendous progress has been made in reducing the amount of human engineering effort needed to produce fast kernels for deep learning, much work remains to be done.
Current approaches have largely focused on a narrow range of dense linear-algebra inspired operators, although it is unclear whether these operators have been chosen simply because of the ease of mapping them efficiently to hardware or because of their suitability to deep learning architectures.
Optimizing arbitrary computation, especially on an end-to-end computation graph remains challenging, as even the subproblem of choosing how to slice the graph is nontrivial.

Perhaps the clearest example of "slicing" the graph is the current dichotomy between "graph optimization" and "kernel optimization."
Current approaches have demonstrated clear benefits by partioning different kernels in deep learning computation graphs.
However, these approaches still consider deep learning kernels as indivisible black boxes to tame the search spaceof possible and valid rewrites.
The development of future primitives is likely hamstrung by these limitations, as even arbitrary ``numpy'' style scripting remains difficult for optimization.

Finally, the need for any human insight into the structure of kernel implementations appears unsatisfying in the presence of ``tabula rasa''~\cite{silver2017mastering} approaches for reinforcement learning.
Even in the absence of template driven approaches, current state-of-the-art tuning methods require a considerable amount of human insight to design the search space~\cite{zheng2020ansor} or promising transformations available to the optimizer.
Ideally, to minimize programmer effort and maximize generalization, a ``tabular rasa'' approach would allow for competitive optimizations without specifying a constrained search space or narrow set of availble transformations tailored to specific architectures ahead of time.
