\section{Introduction}
Images are ubiquitous on the modern web. 
With the rapid expansion of social media services, the largest social media networks now host billions of images~\cite{huang2013analysis}.
Image hosts face the challenge of handling the massive rates at which users upload images, especially as scaling of cost per gigabyte slows~\cite{gupta-mascots14}. 
This issue is compounded by the need to store each image at multiple resolutions to support different contexts or devices. 
In 2010, Facebook stored up to 4 different versions of each image~\cite{beaver2010finding}, later reporting that \emph{dynamic resizing} was also performed~\cite{huang2013analysis}.
\emph{dynamic resizing} saves capacity by generating low resolution copies of images on the fly without committing them to storage.

Faced with a similar problem, Flickr~\cite{flickr} switched to dynamic resizing and reported that doing so helped to eliminate the need for storage capacity upgrades for an entire year. 
While dynamic resizing is an attractive method for reducing storage overheads, it introduces two main trade-offs. 
First, computation is traded for capacity: when an uncached image is requested, the image must be decoded and resized.
Second and perhaps more importantly, bandwidth is traded for capacity: reading the entire source image for resizing can waste bandwidth.
Bandwidth can be precious in cold storage scenarios that sacrifice performance for cost and density~\cite{black2016feeding} or when an access misses in the cache.

This paper proposes repurposing \emph{progressive JPEG} to reduce both read bandwidth and storage overheads.
%
The progressive JPEG standard specifies a variant of JPEG images originally designed for bandwidth-constrained networks.
In a progressive JPEG image, image data is partitioned and arranged by frequency content instead of by vertical position in an image (scanline), allowing for a lossy preview before the entire image has been downloaded.
We demonstrate that by repurposing progressive JPEG, a significant portion of read bandwidth can be saved by reading only the necessary image data for resizing.
Additionally, we show that tuning encode-time parameters to match predefined image sizes can further reduce read bandwidth.

Finally, we characterize the cost of decoding custom progressive JPEG directly on the client relative to decoding resized baseline images. 
We find that the computation--bandwidth trade-off favors transcoding images on the server side, where the computational costs are comparable to a baseline dynamic resizing scheme.
