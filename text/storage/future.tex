\section{Limitations and Future Work}
%\subsection{Limitations}
A limitation of our current implementation is the cost of evaluating custom scan configurations.
Our na\"{i}ve implementation takes days to process 24,988 images on 8 cluster nodes (12 cores/12 threads per node) with Westmere-class CPUs. 
We suspect that this time can drastically reduced without sacrificing significant bandwidth savings by aggressively pruning the search space or applying machine learning techniques to choose scan configurations.
Note that while the customization process for progressive JPEG is currently expensive, it only needs to be performed once, at write time. 

The PSNR metric is limited in its relevance to perceived visual quality~\cite{wang2002no}.
We often found that the PSNR of higher resolution resizes was higher than that of lower resolution resizes even with less image data read---this issue may be mitigated with conservative PSNR thresholds.
To the best of our knowledge, there is no standard, widely used method of computing the image quality of a resized image derived from a source image.
%Ideally, an image quality metric would not describe the similarity or difference of two images but be computable without a reference image.
%Such a scheme would resemble a typical use case more closely as users of an image hosting service do not need a reference image to evaluate image quality.

Finally, an issue when using progressive JPEG for dynamic resizing is the \emph{minimum} resolution of the resized images.
Progressive JPEG is less efficient in terms of read bandwidth for resizes smaller than 10\% of the source image, which may limit savings when the source images are much higher in resolution than their resized versions.
This threshold is due to JPEG's use of $8\times8$ macroblocks: even a single frequency coefficient represents at least $\frac{1}{64}$ of the total image data.
Even when approximations are used, this approach may require more read bandwidth than pre-resized images.
Still, using progressive JPEG should be more space-efficient than baseline JPEG for dynamic resizing.
%\paragraph{Future Work}

We expect that a solution to reduce the cost of enumerating custom JPEG scan configurations will be to prune the search space to a much smaller subset of likely ``good'' configurations.
It may be possible to obtain comparable results by only trying a few custom scan
configurations per image---with this subset being determined by identifying the best configurations when na\"{i}vely re-encoding a larger dataset of images.
Along these lines, even choosing from a larger pool of configurations may be tractable if a machine learning model is applied to each image to choose the best configuration.
